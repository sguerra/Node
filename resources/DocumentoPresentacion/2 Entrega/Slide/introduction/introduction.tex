\section{Introducción}
\frame
{
  \frametitle{Introducción} 
 Las búsquedas de empleo por Internet pueden ser una pérdida de tiempo si no se cuenta con las herramientas adecuadas, además de que la mayoría de las vacantes que se presentan no son actuaes o bien pueden tratarse de estafas.
\newline
Nuestro sistema es una alternativa a la búsqueda de empleo tradicional, las búsquedas electrónicas actualmente facilitan muchas de  las actividades cotidianas, buscar empleo no es la excepción,  "Buscador de Oferta Laboral TI"  será un sistema de  fácil uso pero de gran potencia para la vinculación Empleado-Empleador. 
}
\frame
{
\frametitle{Descripci\'on del funcionamiento del sistema}
	
	\subsection{Descripción del funcionamiento del sistema}

\begin{enumerate}
	\item Las empresas podrán crear perfiles y así mismo vacantes para que los aplicantes puedan solicitar el puesto que más les interese. Además las empresas podrán conocer a los aplicantes, sus capacidades y su CV, y así decidir por el que más les convenza. 
	\item El administrador del sistema podrá administrar a las empresas y a los aplicantes que se hayan registrado en el sistema, y después de investigarlos, darlos de baja del sistema si es que se trata de una empresa que no exista o de algún aplicante que no sea real. 
\end{enumerate}

}

\frame
{
\frametitle{Descripci\'on del funcionamiento del sistema}

\begin{enumerate}
	

	\item El sistema determinara que vacantes y en que empresas esta más capacitado el aplicante dependiendo sus aptitudes y habilidades.
	\item Los aplicantes podrán ver los vacantes en la página principal del sistema de las empresas que se hayan registrado y si los aplicantes ya se encuentra registrados podrán contactar con la empresa y andarles su CV para que estas los revisen. 
\end{enumerate}

}

