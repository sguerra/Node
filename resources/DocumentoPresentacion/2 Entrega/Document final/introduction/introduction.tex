\section{Introducción}
Encontrar un trabajo ya no se limita a enviar CV por correo ni a completar solicitudes de empleo en papel. Gracias a Internet, ahora se puede buscar trabajo en línea. 

\subsection{Marco Teórico }
\begin{itemize}
\item \textbf{Empleo} es el trabajo realizado en virtud de un contrato formal o de hecho, individual o colectivo, por el que se recibe una remuneración o salario. Al trabajador contratado se le denomina empleado y a la persona contratante empleador. [1]
\item \textbf{Empleado} es una persona que desempeña un cargo o trabajo y que a cambio de ello recibe un sueldo.[2]
\item \textbf{Empleador} es, en un contrato de trabajo, la parte que provee un puesto de trabajo a una persona física para que preste un servicio personal bajo su dependencia, a cambio del pago de una remuneración o salario. La otra parte del contrato se denomina trabajador o empleado.[3]
\end{itemize}


\subsection{¿Qué es el comercio electrónico?}
El comercio electrónico es definido por los estudios de la Organización para la Cooperación y el Desarrollo Económicos (OCDE) como el proceso de compra, venta o intercambio de bienes, servicios e información a través de las redes de comunicación. Representa una gran variedad de posibilidades para adquirir bienes o servicios ofrecidos por proveedores en diversas partes del mundo. Las compras de artículos y servicios por internet o en línea pueden resultar atractivas por la facilidad para realizarlas, sin embargo, es importante que los ciberconsumidores tomen precauciones para evitar ser víctimas de prácticas comerciales fraudulentas. [1]




\newpage
\section{Problematica}
Las búsquedas de empleo por Internet pueden ser una pérdida de tiempo si no se cuenta con las herramientas adecuadas, además de que la mayoría de las vacantes que se presentan no son actuaes o bien pueden tratarse de estafas.


\subsection{Propuesta de solución}
Seremos un intermediario que ayudará  a los Empleadores y Empleados a iniciar el contacto, nuestro sistema será capaz de realizar una vinculación por medio información clave como lo es: la profesión, áreas de interés y especialización, experiencia laboral y expectativas económicas.
Somos una alternativa a la búsqueda de empleo tradicional, las búsquedas electrónicas actualmente facilitan muchas de  las actividades cotidianas, buscar empleo no es la excepción,  "Buscador de Oferta Laboral TI"  será un sistema de  fácil uso pero de gran potencia para la vinculación Empleado-Empleador. 

\subsection{Descripción del funcionamiento del sistema}

\begin{enumerate}
	\item Las empresas podrán crear perfiles y así mismo vacantes para que los aplicantes puedan solicitar el puesto que más les interese. Además las empresas podrán conocer a los aplicantes, sus capacidades y su CV, y así decidir por el que más les convenza. 
	\item El administrador del sistema podrá administrar a las empresas y a los aplicantes que se hayan registrado en el sistema, y después de investigarlos, darlos de baja del sistema si es que se trata de una empresa que no exista o de algún aplicante que no sea real.
	\item El sistema determinara que vacantes y en que empresas esta más capacitado el aplicante dependiendo sus aptitudes y habilidades.
	\item Los aplicantes podrán ver los vacantes en la página principal del sistema de las empresas que se hayan registrado y si los aplicantes ya se encuentra registrados podrán contactar con la empresa y andarles su CV para que estas los revisen. 


\end{enumerate}







